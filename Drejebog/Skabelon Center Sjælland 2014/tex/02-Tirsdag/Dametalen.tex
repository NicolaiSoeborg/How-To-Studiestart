Ærede kvinder, mine Damer! \\
Vi er her samlet her i dag som ligesindede, \underline{\textbf{UDEN MÆND}}. Og hvorfor er vi det? Det er vi fordi vi har en historisk pligt til at samles for at praktisere én af den uendelige mængde af ting, som vi kvinder gør bedre end mænd. Og hér snakker jeg ikke om evnen til at dufte godt, være syg uden at klynke, rydde op i køkkenet UNDER madlavning eller utallige andre kompetencer såsom evnen til at koncentrere sig i længere tid. Jeg taler heller ikke om evnen til at lave en matematikaflevering, mens vi taler i telefon og samtidig får klaret dagens knibeøvelser. Nej, jeg snakker om evnen til at diskutere \textbf{VIGTIGE TING!} I år 0 blev det lille Jesus barn født af ingen ringere end selveste Jomfru Maria; En fantastisk kvinde, hun klarede det hele selv, \underline{\textbf{UDEN MÆND}}. Men hvordan gjorde hun det? Jo, hende og Gud havde diskuteret \textbf{VIGTIGE TING!} Mændenes historie fyldt med blod, og meningsløs vold og død. Det var aldrig gået så galt, hvis blot vi havde levet \underline{\textbf{UDEN MÆND!}} Eksemplerne er talrige: Alexander den Stores erobringstogt i år 336 f. kr., Hunnerkongen Attilas togt i 430 med plyndring og voldtægt, Vikingerne, Napoleon, Stalin, Hitler. Alle disse timer og liv spildt, blot fordi mænd ikke kan finde ud af at diskutere \textbf{VIGTIGE TING!} Snakker vi derimod kvindelige regenter, så har der, for hver eneste konge igennem historien været en dronning ved hans side, men når man kigger nøje efter vil man opdage, at der ikke for hver dronning har været en konge! Jeg taler om Kleopatra, Elizabeth I og selvfølgelig om Danmarks to Margrether’; Dronninger som har formået at samle Norden, lede Nationen og regere over folket og de har gjort det \underline{\textbf{UDEN MÆND}}. Men hvordan gjorde de det? Ved at diskutere \textbf{VIGTIGE TING!} På trods af, at mændene dræber og kvinderne skaber, så er det alligevel mændenes bedrifter som har fået størst genklang i historien. Derfor vil vi i dag gerne ære to kvinder, som er blevet overset, men som har banet vejen for, at vi sidder her i dag, nemlig: Agens Klingberg og Betzy Meyer, der i 1897 blev Danmarks første kvindelige ingeniører. Det gjorde de \underline{\textbf{UDEN MÆND, PÅ TRODS AF MÆND}} og de gjorde det ved at diskutere \textbf{VIGTIGE TING!} Når mænd mødes, uden kvinder, så bruger de tiden på at diskutere ting såsom, hvis prut der lugter mest, hvem der er kommet længst i Command and Conquer, hvem der har håneretten i ølrisk eller hvem der er den bedste James Bond. Er det vigtige ting? (\textbf{NEJ!}) Vi kvinder sidder IKKE og klør os i skridtet, piller næse og finder navlefnuller uden at kunne overskue at rejse os op for at hente en bajer, \textbf{NEJ!}, når kvinder mødes, \underline{\textbf{RIGTIGE KVINDER}}, nøjagtigt som vi gør her i aften og som vi vil mødes i al fremtid, bliver relationer skabt, verdenssituationen vendt, problemer løst, skønhed udforsket og viden delt! Der bliver sammenlignet outfits, snakket om penislængder, og analyseret sms’er. Men først og fremmest så bliver der diskuteret.. \textbf{VIGTIGE TING! VIGTIGE TING! VIGTIGE TING!}
