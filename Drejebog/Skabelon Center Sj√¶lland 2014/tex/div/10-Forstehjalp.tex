\section*{Almindelig førstehjælp på Rustur:} Her er en liste over ting I kan komme ud for på rusturene:

\begin{itemize}
\item \textit{\textbf{Fremmedlegemer i luftvejende}}: Ved delvist blokeret (patienten kan godt sige noget) opfordres der til hoste og evt. slag med flad hånd mellem skuldrebladene, samtidig med hoste. Ved total blokering anbefales det, at der veksles mellem at give 5 slag mellem skuldrebladene og 5 hårde stød i bughulen op mod mellem gulvet, Heimlich (armene rundt om patient der bøjer sig forover).\\
\item \textbf{\textit{Hjernerystelse:}} Kommer til udtryk ved en eller flere af følgende symptomer: Hovedpine, Kvalme, Opkast, Svimmelhed, Synsforstyrrelser, Træthed og tap af korttidshukommelse. Førstehjælpen er: Læg personen for at slappe af/sove og hold tæt observation i mindst 24 timer, f.eks. ved at vække personen en gang i timen. Dette er for at sikre, at tilstanden ikke forværres. Bliver personen tiltagende dårlig, skal der søges skadestue eller læge hurtigst muligt. \\
\item \textbf{\textit{Forbrænding}}: Mindre forbrændinger skal straks køles ned med, så koldt vand som patienten kan holde ud, indtil det ikke gør ondt længere. Sværere forbrændinger (2. grads kan ses ved at der er vabler/sår og 3. grads vil huden være lederagtig og blødende) skylles med tempereret vand (ca. 18-19 grader), der alarmeres og skylning fortsætter under transport til videre behandling. \\
\item \textbf{\textit{Hedeslag/solstik}}: Symptomer: hovedpine, svimmelhed og træthed. Huden vil være varm og lyserød. Patienten vil svede og være konfus. Førstehjælpen er at få patienten i skygge og fjern/løsne tøj. Læg koldt omslag ved pande, nakke, håndled, lyske og ankler, giv noget koldt at drikke. \\
\item \textbf{\textit{Knoglebrud}}: Ved ben- eller bækkenbrud tilkald ambulance og støt bruddet i findestillingen. Ved mindre brud kan man køre selv og holde bruddet i ro, evt med trekant tørklæde. \\
\item \textbf{\textit{Forstuvning}}: Opstår ved for stor og forkert belastning af et led, symptomerne kan afhjælpes ved at benytte sig af {\color{red}\textbf{R.I.C.E}}-princippet:
\begin{itemize}
\item[{\color{red}\textbf{R}} =] Rest. Hold det beskadigede område i ro i det første døgn.
\item[{\color{red}\textbf{I}} =] Ice. Afkøl med is(køleposer kan bruges) eller koldt vand. Køl i ca. 30 min. og hold derefter en times pause. Fortsæt dette til hørelsen er aftaget.
\item[{\color{red}\textbf{C}} =] Compression. Sørg for at lægge pres på det beskadigede sted øjeblikkeligt, hold det i ca. 10-20 min og anlæg derefter et støttebind. 
\item[{\color{red}\textbf{E}} =] Elavation. Løft det beskadigede område over hjerte højde. \\
\end{itemize}
\item \textbf{\textit{Snitsår}}: Rens med renseservietter og sæt plaster på, brug evt. lille kompresforbinding. Løft for at standse blødning.\\
\item \textbf{\textit{Næseblod}}: Pres 2 fingre på næsen og bøj hovedet let forover. Læg evt en kold klud eller is over næsen, for at få blodkar til at trække sig sammen. Hvis blødningen ikke er stoppet efter en halv time så søg læge eller skadestue. \\
\item \textbf{\textit{Flåt}}: Fjern flåten med flåt-tang. Mas aldrig på kroppen af flåten. Kan den ikke fjernes, søg læge. \\
\item \textbf{\textit{Bistik}}: Hvis man bliver stukket i  mund eller hals skal der søges læge eller skadestue, er dette langt væk, tilkald ambulance. Normale bistik kan afhjælpes med en sukkerknad, der trækker giften ud, husk at se om brodden er kommet ud. \\
\item \textbf{\textit{Astma}}: Ved anfald, støt personen i at sidde eller stå. De skal ikke ligge ned, da dette kan forværre åndenøden. Armene over hovedet kan også hjælpe. Sørg for frisk luft og giv psykisk førstehjælp. Evt. hjælp med personens egen medicin eller astmaspray. \\
\item \textbf{\textit{Diabetes}}: Skyldes nedsat eller ingen produktion af hormonet insulin, der transporterer sukker fra blod ind i cellerne. Anfald kan skyldes lavt eller højt blodsukker. Ved lavt blodsukker gives noget sødt at spise eller drikke. Ellers trinvis førstehjælp. Er man i tvivl om det er højt eller lavt blodsukker så giv noget sødt alligevel, da det ikke gør den store forskel, hvis blodsukkeret allerede er for højt. Man må ikke selv give insulin til sukkersygepatienter. Hvis de er ved bevidsthed kan man hjælpe, de har som regel selv helt styr på det. \\
\item \textbf{\textit{Epilepsi}}: Få personen til at ligge ned og løsne stramt tøj om halsen. Beskyt hovedet mod stød, når krampen ophører lægges personen i stabilt sideleje for at skabe frie luftveje. De vil oftest være meget trætte. Forsøg ikke at stikke noget mellem tænderne, men forhold dig i ro.
\end{itemize}