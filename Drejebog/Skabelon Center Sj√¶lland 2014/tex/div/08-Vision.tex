\section{Visionsoplæg}
\subsection*{Generelt:}
\underline{\textbf{Formål og retningslinjer:}}\\
Hvad er formålet med rusturen?\\
Russerne skal hinanden at kende og skal ud af deres comfort zone. Få skabt et netværk på hele DTU på tværs af retningerne. 

\underline{\textbf{Holdning til rygning?}}\\
Ingen rygning indenfor. Vi sætter en bænk og spande op, alle skodder skal i spande. 10-sekunders reglen tæller også for skodder. Venlig påmindelse til russer der ikke kan finde ud af det. Opfordr russerne til at spørge om det er ok at ryge i grupperne. Hvis der er skodder over det hele strammes reglerne. 

\underline{\textbf{Holdning til alkohol?}}\\
\textbf{Russer:} Ingen medbragt alkohol. Alt skal indleveres, så får de det tilbage bagefter. Vi holder øje med hvor meget de drikker og kan give for fulde russer vand fra køkkenet. Det er primært ædru-ansvarlige der vurderer om folk er for fulde. Hvis der er nogen der bliver for fuld og til grin, er det en god ide at gribe ind og evt. snakke med de ``højtråbende'' russer.\\
\textbf{Vektor:} Vi har en flaske shots på vektorværelset som man må drikke af hvis man trænger til at få festen startet og ikke kan overskue øl. Man bliver ikke på værelset og safter sig ned, men tager et shot og går ud igen. Vi kan tage et shot til kaosmøderne.\\
Køkkenhold: Har relativt frie tøjler, så længe de kan administrere det og lave god mad til tiden og kan deltage i de aktiviteter, de skal med til. Tager en romtønde med samt fernet. Russerne får klapper for øjnene og en sjat rom til opvasken. Skal en rus have et shot må man gerne lige advare køkkenet på forhånd.

\underline{\textbf{Hash og euforiserende stoffer:}}\\
Nul tolerance. Vi giver dem chancen for at indlevere eventuelt medbragte sager i starten, hvorefter det skylles ud i toilettet med det samme. Dagsansvarlig kigger på mens russen skyller det ud. Hvis vi kan lugte noget, fortælles det til næste samling at det er hjemsendelsesgrund hvis man fanges i at ryge. Hvis stoffer bliver et omfattende problem på turen skal der ringes til Inge, og alle sendes hjem. Er der tale om hårde stoffer (andet end hash), indkaldes der til hastekaosmøde (prik på skulderen) og der ringes til politiet, og de to ædru og russen møder politiet ude ved vejen, så hele rusturen ikke opdager så meget.

\underline{\textbf{Nøgenhed:}}\\
Vi opfordrer ikke til nøgenhed og må ikke selv være nøgne. Smider russerne tøjet bedes de pænt men bestemt om at tage tøj på igen. Gælder også for mankinis. 

\underline{\textbf{Må våben medbringes? Hvad gøres der?}}\\
Nej. Der ringes til politiet. Vi gør opmærksom på det i invitationen. 

\underline{\textbf{Udflugter fra hytten?}}\\
\textbf{Med vektor:} Hvis nogen har lyst vil vektorer gerne løbe en tur om morgenen. Har man overskud kan man gå en tur med nogen.\\ 
\textbf{Uden vektor:} Hvis man gerne vil have en løbetur alene skal man være ædru, og dagsansvarlig kontaktes. Man skal have telefon med. Grunden må ikke forlades fuld uden vektor.

\underline{\textbf{Mobiltelefoner?}}\\
\textbf{Vektorer:} Har for så vidt muligt mobil på sig. Begge ædruveste har altid en opladt telefon på sig. Sørg for at den er ladet op. KABS har mobil på sig hele tiden. Nul sociale medier på turen. Skal man tjekke facebook kan man gøre det inden man går i seng. Slå app-notifikationer fra, så man ikke kommer til at kigge på telefonen hele tiden. Højdere er smart. Vi skaffer en ædrutelefon, som folk kan ringe til og dagsansvarlig har på sig.\\
\textbf{Russer:} Vi opfordrer alle russer til at lade mobilen ligge i tasken, da de ikke er på rustur for at skrive med vennerne. Opkald er ok i nødstilfælde.

\underline{\textbf{Kamera?}}\\
Vi tager et kamera med på turen, russer må gerne tage billeder. Vi laver censur på billederne, der lægges op i den lukkede gruppe på facebook. Vi holder øje med om russerne er for fulde til at håndtere det. Kørselsansvarlig har kameraet.

\underline{\textbf{Håndtering af naboer?}}\\
En vektor tager rundt med telefonnummeret på ædrutelefonen, som de kan ringe til hvis der bliver problemer. 

\underline{\textbf{Håndtering af hyttefar?}}\\
Hyttefar likes, vi gør som han siger. Det er ædruansvarlig der snakker med hyttefar.\\
\textbf{Skader på hytten:} Tag billeder af skader ved ankomst og meld det til hyttefar med det samme. Det samme gælder mangler.\\
\textbf{Uheld:} Der ringes til hyttefar hvis det er seriøst. Russer opfordres til at sige det med det samme, hvis der sker noget. Det hele noteres.\\
\textbf{Hærværk:} Russen betaler selv, og hvis det er seriøst kan der tages stilling til om der skal være yderligere konsekvenser til et kaosmøde.

\subsection*{Russer:}
\underline{\textbf{Gruppepres/indpakning af russer i vat?}}\\
Vi griber ind med liking og andre aktiviteter hvis nogen udsættes for gruppepres. Det er også voksne mennesker, som selv må sige fra nogle gange, så vi skal ikke pakke dem ind i vat. Vi må godt presse dem ud af deres comfort zone, hvis de eksempelvis helst ikke vil være med i en leg. 

\underline{\textbf{Hvad gør vi ved en vektorlegende rus?}}\\
Hvis det ikke bliver for meget må de gerne hjælpe til. Vi bestemmer stadig hvordan tingene forløber. 

\underline{\textbf{Uudholdelige russer?}}\\
En vektor der hverken har vest eller er retningsvektor er bad guy og fortæller stille og roligt russen, at han gør noget som mange synes er irriterende, og at han nok skal stoppe med det. Dette tages først op til et kaosmøde, så alle ved hvad der sker. Virker det ikke, er oversavning altid en mulighed. 

\underline{\textbf{Opmærksomhedskrævende rus?}}\\
Samme situation som ovenfor. Pas på med at save personen over, da det kan eskalere situationen.

\underline{\textbf{Rus, der lægger an på vektor}}\\
Hvis en vektor ser det, gør man den pågældende vektor opmærksom på problemet. Er det et tilbagevendende problem kan man trække russen til side og forklare reglerne for samkvem i studiestarten. Er man i knibe, kan man råbe ananas. 

\underline{\textbf{Russer, der er sammen internt}}\\
Ja tak! Så længe det ikke bliver påtrængende over for de andre. Folk må gerne knalde på værelserne, men ikke andre steder. Det giver badges. Hvis russen beder om ikke at få det offentliggjort til morgenrejsning er dette ok. 

\underline{\textbf{Udlevering af medicin til russer?}}\\
Hvis folk virkelig har brug for en panodil, kan dagsansvarlig udlevere. Man skal holde øje med at folk ikke får for mange. 

\underline{\textbf{En rus med egen bil på turen?}}\\
Inddrager nøglerne indtil turen er slut. Hvis personen mangler noget, kan kørselsansvarlig køre efter det, og det skal aftales på forhånd hvis personen skal køre. Nul tolerance hvis han har drukket. 

\underline{\textbf{Aktivering af stille russer}}\\
Prøv at komme ned på deres niveau, sid og snak med dem og forsøg at få dem med i et spil. Måske ikke et drukspil, men bare en gang Bezzerwizzer eller 500. Prøv at få flere med i spillet, så russen bliver inddraget i en gruppe.

\underline{\textbf{Rus der gerne vil hjem:}}\\
Retningsvektor og evt. tværvektor snakker med russen for at høre om problemet og forsøger at like og overtale russen til at blive. Er det helt skidt tages det op til et haste-kaosmøde. Skal personen hjem, køres russen til nærmeste station, såfremt det er forsvarligt (og russen ikke dør). 

\underline{\textbf{Voldelige russer?}}\\
Dagsansvarlig kontaktes. Der skal mindst to vektorer med. Det kan være farligt at frembruse for offensivt, ofte er det bedre at det er piger der stopper en slåskamp. Tag fat i den ene rus og snak med ham inden det eskalerer og lad situationen køle af. Lad være med at tage side. Få de andre russer væk hvis der opstår en situation. Hvis det går helt galt kan politiet kontaktes.

\underline{\textbf{Kampdrikkende russer:}}\\
Prøv at fortælle dem at der er aktiviteter i løbet af dagen og det vil være smart at holde lidt igen indtil aftensmaden. Fortæl at det er sjovest at være i stand til at deltage i alle aktiviteter. Retningsvektoren kontaktes, så han/hun kan snakke med russen. Er der en der er meget ramt om morgenen skal de med op til morgenrejsning, men de kan få lov til at få en lur hvis de virkelig trænger. 

\underline{\textbf{Klæbende rus:}}\\
Prøv at få involveret russen i aktiviteter med andre russer ala den stille rus. Man kan godt opfordre dem til at snakke med de andre russer. Alternativt kan man smide russen af på en anden vektor.

\underline{\textbf{Opsyngning:}}\\
Man synger ikke enkelte personer op. Hold det på et minimum til morgenmad (og frokost), og lad være med at få det til at eskalere under aftensmaden. Ædruansvarlig kan gå rundt til de andre vektorer og bede om ro hvis det stikker for meget af. Man må ikke bare overdøve et hold med ``man drikker hvis man ikke kan synge''.
\subsection*{Interne problemer:}
\underline{\textbf{Vektor er sammen med rus:}}\\
Vi hiver vektoren og russen til side og forklarer, at det var en fejl men det ikke var russens skyld. Vektoren er ædru resten af turen, og KABS finder ud af om der skal ske yderligere konsekvenser. Der uddeles ikke scorebadges.

\underline{\textbf{Fuld under/før ædruvagt:}}\\
Ædru resten af turen. Man får en ny vagt. Den mindst fulde bliver hurtigt ædru og overtager. Den resterende ædruvest får begge veste. 

\underline{\textbf{Uenighed:}}\\
Dagsansvarlig har sidste ord. Flyt diskussionen ind på vektorværelset så vi ikke står og diskuterer foran russerne. Lad være med at rette på folk foran russerne. Hvis man bliver uvenner snakker man med KABS om problemet. Det er en god ide lige at få sovet på det og så få snakket om det dagen efter. 

\underline{\textbf{Dovenskab:}}\\
Find dagsansvarlig eller KABS. Der skal likes. Hvis man har brug for en powernap spørger man lige dagsansvarligt.  

\underline{\textbf{Andres ansvarsposter:}}\\
Vi stoler på at de ansvarlige har styr på deres ting, og at man spørger om hjælp hvis man har brug for det. 

\subsection*{Interne forventninger og aftaler}
\underline{\textbf{Forventninger til madholdet?}}\\
God mad til tiden og rom og kaffe og fernet tak. Og kolde øl hvis der er plads. Kan hjælpe til med løb. Tager gerne imod russer der har brug for en pause, hvis de bliver advaret på forhånd. Tidsændringer meldes til køkkenet i god tid. Dagsansvarlig holder styr på det. Der kommer en pirat med til alle kaosmøder. Forsinkelser i køkkenet meldes også i god tid.  I køkkenet bestemmer piraterne. Vil gerne have drejebog.

\underline{\textbf{Hvad forventes der af ædruvagter?}}\\
Begge veste skal kunne udgå fra løb uden problemer, hvis der skulle opstå noget.\\
\textbf{Dagsansvarlig:} Helst være ædru tak. Har de endelige beslutninger. Holder styr på at tidsplanen overholdes i løbet af dagen. Har kontakten til køkkenet. Indkalder til kaosmøder og har styr på dagsordenen. Holder orden til måltiderne og giver beskeder eller får en med en god stemme til det. Har fuldstændig styr på hvad der skal ske i løbet af dagen. Har kontakt til den anden vest. Husk at spørge om hjælp og uddeleger.\\
\textbf{Kørselsansvarlig:} Ædru, kan køre bil. Skal være frisk nok til at køre bil, så man må gerne tage en lur. Giver besked til dagsansvarlig hvis man tager en lur. Bunder når vesten lægges. Hjælper dagsansvarlig hvor der er brug for det. Tager billeder!\\
\textbf{Overdragelse af veste:} Bunder en øl/cocio, men styrer alligevel sin brandert. 

\underline{\textbf{Hvad er jeres ansvar og forventninger som vektorer til hinanden:}}\\
Man saver sig ikke ned så man ikke kan tage ansvar. Ølbowling inden ansvar er en dårlig ide. Der er hovedansvarlige for alle ting, men alle skal vide hvad der sker på turen. Vi bakker hinanden op og hjælper hvis vi ser der er brug for det. Man er stadig vektor selvom man ikke har vest på.  

\underline{\textbf{Forholdet mellem sjov og seriøse ting:}}\\
Der kommer til at være lidt af hvert. 

\underline{\textbf{Førstehjælpskassens placering:}}\\
I køkkenet.

\underline{\textbf{Forholdet mellem tid i retnings og tværholdet:}}\\
Primært tværholdet, da man er sammen med retningen resten af året. Ingeniøropgaven bliver i retning, og måske et løb. 

\addcontentsline{toc}{subsection}{Kompetencelisten}

\underline{\textbf{Hvem er gode til hvilke situationer:}} Bræk, blod, rolig tale, førstehjælp osv.?\\

\begin{table}[H]
\begin{tabu}{l *{8}{|c}}
\specialrule{1pt}{0pt}{2pt} \rowfont{\bfseries}
Hvem kan hvad       & Stive & Hæmor & Mighty & Buddha & Randi & Karla & Clint & Farav \\ \specialrule{1pt}{2pt}{1pt}
Bræk                & Ja    & Ja    & Ja     & Ja     & Ja    & Nej   & Nej   & Ja    \\ \specialrule{.25pt}{1pt}{1pt}
Blod                & Ja    & Nej   & Ja     & Ja     & Ja    & Ja    & Nej   & Ja    \\ \specialrule{.25pt}{1pt}{1pt}
Knogler             & Nej   & Nej   & Måske  & Ja     & Nej   & Måske & Nej   & Ja    \\ \specialrule{.25pt}{1pt}{1pt}
Rolig tale          & Ja    & Ja    & Ja     & Ja     & Ja    & Ja    & Ja    & Ja    \\ \specialrule{.25pt}{1pt}{1pt}
Taler generelt      & Ja    & Ja    & Ja     & Ja     & Ja    & Måske & Ja    & Ja    \\ \specialrule{.25pt}{1pt}{1pt}
Stoffer (påvirkede) & Ja    & Måske & Måske  & Ja     & Måske & Måske & Måske & Nej   \\ \specialrule{.25pt}{1pt}{1pt}
Aggressive russer   & Ja    & Ja    & Ja     & Ja     & Måske & Ja    & Nej   & Ja    \\ \specialrule{.25pt}{1pt}{1pt}
Ældre               & Ja    & Nej   & Nej    & Ja     & Ja    & Ja    & Ja    & Ja    \\ \specialrule{.25pt}{1pt}{1pt}
Partystarter        & Ja    & Ja    & Ja     & Ja     & Måske & Nej   & Ja    & Ja    \\ \specialrule{.25pt}{1pt}{1pt}
Ædrubonding         & Ja    & Måske & Semi   & Ja     & Ja    & Måske & Ja    & Nej   \\ \specialrule{.25pt}{1pt}{1pt}
Minimal søvn        & Ja    & Ja    & Ja     & Ja     & Ja    & Nej   & Måske & Ja    \\ \specialrule{.25pt}{1pt}{1pt}
Førstehjælp         & Ok    & Ok    & Ok     & Ja     & Ok    & Ja    & Nej   & Ja    \\ \specialrule{1pt}{1pt}{0pt}
\end{tabu}
\end{table}

\underline{\textbf{Rus lægger an på vektor - hvordan stoppes det?}} (sang, bestemt sætning e.l.?)\\
Kald på en vektor i nærheden. Hvis man ser det, kan man hente vektoren udenfor. Ananas og kigge på stjerner. 

\underline{\textbf{Hvor mange vektorer er vågne pr rus? Hvem må lukke festen?}}
Er det ok at 2 vektorer/KABS er vågne og holder styr på festen, hvis 30 russer er oppe?\\
Vektorerne lukker og slukker, selvom der er 10-15 stykker der stadig gerne vil feste. Sig at vi også har et program for morgendagen. Musikcomputeren kan bare slukkes. Der skal være mindst to vektorer vågne uanset antallet af russer. Dagsansvarlig skal godkende at vektorerne går i seng. Medmindre dagsansvarlig sover, i så fald vendes det bare blandt de resterende vektorer. Dagsansvarlig giver ``ansvaret'' til den mindst fulde vektor. Det kan også aftales til kaosmøder. 

\underline{\textbf{Hvornår sendes folk i seng?}}\\
Kl. 4 senest. 

\underline{\textbf{Søvn - hvor og hvor meget må vi sove?}}\\
Dagsansvarlig skal sørge for at have sovet så meget ud som muligt inden sin vagt. Ellers spørger man dagsansvarlig om man må tage en powernap i løbet af dagen. 

\underline{\textbf{Kaosmøder - hvordan og hvor ofte?}}\\
Morgenmøder er en god ide. Vi ser på den overordnede plan for at vurdere om vi skal have et møde senere på dagen også. Dagsansvarlig gennemgår hvad der skal ske i løbet af dagen, og de ting der er sket tages op. Der findes ud af hvem der skal have hvilke badges og hvad der skal læses op af sladderkassen. 

\subsection*{Generelle regler:}

\underline{\textbf{Alkoholpolitik - vektorer}}\\
Når vagten indtræder, skal den ansvarlige være 100 \% ædru. Man skal senest
stoppe med at drikke, 16 timer får vagten indtræder. Alle ansvarlige skal
stoppe med at drikke torsdag midnat.

\underline{\textbf{Autoritet}}\\
Tilegnede refleksveste skal bæres under hele vagten. Ædruvagt har ALTID sin
mobil på sig.

\underline{\textbf{Overlevering}}\\
Før en vagt afsluttes, skal en overlevering til de efterfølgende kørsels- og
dagsansvarlige ske. Sker der noget under overleveringen, som kræver kørsel,
kører den daværende kørselsansvarlige.

\underline{\textbf{Drejebogen}}
Hvis man mister sin drejebog og en rus finder den, skylder man en øl til
russen. Hvis man mister drejebogen helt, og den ikke bliver fundet på turen
skylder man en kasse til vektorerne. 

