\section{Velkomsttale}
\subsubsection*{\textbf{Ansvarlig:} \Farav }

Mine damer og herrer - velkommen til Center Sjælland!\\
Vi er nu trådt ind i år 2033. De sidste 20 år er der blevet kæmpet en dødelig krig over klodens sparsomme ølressourcer. Verdenen har været vidne til mange ubehagelige ting, og kun 8 lande er tilbage. De er:

\begin{itemize}
  \item Skotland
  \item Sydøstasien
  \item USA
  \item Ølympen
  \item Brasilien
  \item Vikingeland
  \item Den australske outback
  \item Egypten
\end{itemize}

Vi er samlet på dette sted for at finde den retmæssige verdenshersker, og for at fordele de resterende landområder imellem os. Og den eneste rigtige måde at gøre det på, er naturligvis ved at spille RISK!

I kampen om landene og de sidste ølressourcer vil i naturligvis få hjælp fra jeres vektorer og mig. Mit navn er Farav Ikmer, og jeg er KABS, hvilket står for Koordinator af Bachelor Studiestarten, og jeg læser Fysik og Nanoteknologi på 5. semester. Jeg er naturligvis lederen af Egypten.

\vspace{12pt}
\textbf{Præsentation af vektorer. Sig navn, retning, semester og lande\\
\clint \buddha \stive \hemorides \randildo \karla \mighty 
}\vspace{12pt}

I undrer jer uden tvivl over, hvorfor vi ikke skal have delt verdenshavene imellem os. Dette skyldes at de 7 fugtige områder bevogtes af tre vederstyggelige pirater, som tilfældigvis også har tilbudt at lave mad til os:

\vspace{12pt}
\textbf{Præsentation af hyttebumser. Sig navn, retning og semester\\
\Hyttebums{Pirates}
}\vspace{12pt}

I har sikkert bemærket at Karla og jeg har smukke neongule veste på. Det betyder at vi er ædru og har ansvar. Jeg er dagsansvarlig og skal sørge for at rusturen kører på skinner, og Karla er kørselsansvarlig og kan til enhver tid køre, hvis der bliver brug for det. Disse veste går på skift hver dag, og hvis I har problemer, kan man altid hive fat i en med vesten. Derudover har kørselsansvarlig et kamera, som alle med normaltfungerende balancesans meget gerne må låne.

\subsection*{Regler}
\begin{itemize}
  \item \textbf{Holdkæft-håndtegn} - Når hånden rækkes op tier man stille og rækker selv hånden op. Ingen snakker med hånden oppe.
  \item \textbf{Rygning} - Foregår udendørs. Skodder i spanden. Husk at vise hensyn hvis I ryger i en gruppe med ikke-rygere. 	
  \item \textbf{Smadrede ting} - Hvis man smadrer ting kontaktes dagsansvarlig.
  \item \textbf{Dansk lovgivning} - Gælder! Ingen stoffer, vold og hærværk. Hvis i har noget med så skal det afleveres i dag.
  \item \textbf{Drejebøger} - Hvis I finder en, så giv den til vektoren, der har tabt den - vedkommende kvitterer så med en øl. Lad være med at kigge i den, da det ødelægger turen.
  \item \textbf{Mobiltelefoner og ure} - skal blive i tasken - vi er her for at lære hinanden bedre at kende, og facebook er der også når I kommer hjem.
  \item \textbf{Badekar} - hvad er det? Tag bare, men husk at lægge ting på plads og passe på tingene. 
  \item \textbf{Musikcomputer} - Sæt i kø, ingen øl ved computeren. Det koster en øl at blive taget. Ikke pille ved volumen. Sæt ikke den samme sang i kø flere gange.
  \item \textbf{Badges} - finder man et, afleveres det til rette vedkommende, som så kvitterer med en øl
\end{itemize}

\subsection*{Praktisk info omkring hytten/turen}
\begin{itemize}
  \item Hvor sover man? (Ingen fest på sovesale)
  \item Hvor spiser man?
  \item Hvor skider man?
  \item Hvor leger man?
  \item Hvor går grunden til? (Gør det klart, at man ikke må forlade området)
  \item Ingen adgang skilte
  \item Vækning og morgenrejsning - sidste hold får toilettjansen!
  \item Tjanseskema og aktivitetsskema
  \item Førstehjælpskasse i køkkenet, skaf en vektor hvis der sker noget
\end{itemize}

\subsection*{Køkkenet}
\subsubsection*{\textbf{Ansvarlige:} \Hyttebums{Piraterne}}
\begin{enumerate}
  \item Hyttebumserne bestemmer alt
  \item Reglerne i køkkenet, madplanen, rushjælp til mad mm.
  \item Hvor man afleverer viskestykker / toiletpapir
  \item Vask hænder!
  \item Promiller i køkkenet
  \item Fodtøj
\end{enumerate}

\subsection*{Missionskort og RISK}
\subsubsection*{\textbf{Ansvarlig:} \Mighty}
Hver dag vil hvert hold få udleveret nogle missionskort, hvorpå en mission står skrevet samt hvor mange armeer den er værd (antal stjerner). Når man drager ud på en mission er det vigtigt man hiver fat i en voksen (vektor eller KABS), som kan godkende den forhåbentlig succesfulde mission. Armeerne benyttes til at erobre lande ved at holdet tager sine armeer og placere dem på de lande de vil erobre. Et land er først erobret, når holdet har flest armeer i pågældende land, dog må man ikke erobre et andet holds hjemland. Undervejs vil der også være drabelige kampe mellem holdene som også giver armeer. Det hold som tilsidst har erobret flest lande udnævnes som verdensherskere og vindere af RISK-spillet. \Hashtag{nemt} \\


\subsection*{Øl regler}
\subsubsection*{\textbf{Ansvarlig:}  \stive \hemorides \farav \mighty \buddha \randildo}
\begin{itemize}
  \item \textbf{Drik mellem mærkenrne} - \stive forklarer: når man åbner en øl er man tørstig, derfor er det selvfølgelig vigtigt, at man tager en ordentlig slurk af den lige til mellem mærkerne. Hvis det skulle ske at man drikker mere end mellem mærkerne er man tydeligvis meget tørstig, derfor kan man ligeså godt bunde. Hvis det omvendte skulle ske og man ikke kan drikke nok til mellem mærkerne er man ikke tørstig nok, derfor er det bedst for alle man bunder og starter forfra. Han demonstrere naturligvis undervejs og bunder!
  \item \textbf{Ølsjusk} - \hemorides forklarer og kommer til at spilde sin øl: man spilder naturligvis ikke sin øllebajser/cider, men hvis det skulle ske har man tydeligvis ikke drukket nok og derfor bunder man.
  \item \textbf{Buffalo} - \farav forklarer: tilbage i det vildeste vesten var en meget kendt cowboyder Bufallo Bill: kendt for sine vilde evner med en pistol og sin store kærlighed til øl og whiskey. Dengang i det vilde vesten sad pistolen altid ved højre hofte, så man hurtigt kunne komme til den og skyde hvem der nu var lidt træls. En dag Buffalo sad på hans stam saloon og havde fået serveret sin yndlings whiskey, en yderst exceptionel extra dry single malt whiskey fra Skotland, han sippede til med HØJRE HÅND. Uheldigvis for Buffalo braste en bandit ind af døren i samme øjeblik og skød vildt og voldsomt omkring sig. Buffalo forsøgte at trække sin pistol, men var simpelthen for langsom og nåede det ikke før en kugle gennemborer hans hjerte. Derfor drikker man ALTID med venstre hånd for at kunne nå sin pistol og forsvare sig, dog bunder man ALTID med højre, da man så kan benytte drikkegenstanden til kasteskyts og forsvare sig på den måde. Så hvis man skulle få øje på en person som drikker med højre hånd eller bunder med venstre råber man "BUFFALO!" og personen bunder en genstand.
  \item \textbf{Værktøjsregel} - \mighty forklarer værktøjsreglen mens Hæmorides prøver at åbne en øl, han failer hvorefter Mighty tager over og åbner. Hæmorides bunder og starter forfra. Værktøjsreglen: man har 3 forsøg til at åbne en øl med et hvilket som helst værktøj, man kan også udfordrer hinanden til at åbne en øl med et bestemt værktøj. Hvis der har været flere der har forsøgt køber de en genstand og bunder selvfølgelig også.
  \item \textbf{10 sekunder} - \buddha forklarer: man ønsker ikke at ens øl/cider skulle nå at blive flad, derfor har vi den vigtige 10 sekunders regel. Man har 10 sekunder til at finde kapslen og drikke mellem mærkerne. På den måde passer vi også på miljøet og slipper for at der flyder kapler over det hele. Miljøster passer selvfølgelig ekstra godt på miljøet og har derfor kun 5 sekunder til at finde kapslen.
  \item \textbf{Divine} - \buddha ser presset ud og \randildo diviner, samt forklare reglen. Personen man diviner skal give lov osv.
\end{itemize}

\textbf{HUSK} russerne på at druk er en gentlemans sport, alle skal have det sjovt!

\subsection*{Introduktion til ølsystem}
\textbf{Ansvarlig:} \Mighty
\begin{itemize}
  \item Russer får udleveret stregkodekort af deres vektorer
  \item Aflever fundne stregkodekort – belønning som sodavand/øl
  \item Hvordan skannersystemet virker
  \item Betaling
  \item Svind
\end{itemize}

\subsection*{Udfordrertrøjen} 
\begin{itemize}
  \item \mighty udfordrer \stive i at bunde en øl. Vinderen får udfordrerkappen og viser, hvad man skal skrive.
  \item Tusser findes i spisesalen
  \item Hvad der skal stå: \textbf{Navn, retning (forkortet) samt udfordring}
  \begin{itemize}
    \item Eksempelvis: Stive Irwin (Biotek) Master of Ølbunding
  \end{itemize}
\end{itemize}

\subsection*{Afrunding}
Og det var så krigens regler. I løbet af ugen vil I blive udsat for en masse sjove aktiviteter, og vi vil få besøg fra DTUs rektor og dekan, Polyteknisk Forenings bestyrelse samt IDA. Husk at I altid kan gå til dagsansvarlig eller en anden vektor hvis I har et spørgsmål. I kan nu følge med jeres vektor op til jeres værelser, og herefter er det tid til at få skabt noget nationalitetsfølelse og lavet nogle flag.


