\section{(Sø)Stjerneløb}

\begin{tabular}{L{8cm} L{8cm}}
\textbf{Post 1}                         &  \textbf{Post 4}                              \\
PF, struktur, faglige råd og udvalg     & AUS/Studievejledningen og Fremdriftsreform    \\
Ansvarlig: \Lucyfar                     & Ansvarlig: \Ora                               \\
Sted: Rusrum 2                          & Sted: Rusrum 4                                \\
                                        &                                               \\
\textbf{Post 2}                         & \textbf{Post 5}                               \\
Kollegier, PKS og udlandsophold         & Puma                                          \\
Ansvarlig: \KABS                        & Ansvarlig: \Johnny                            \\
Sted: Rusrum 3                          & Sted: Festsalen (\#PumaUdenMusikVirkerIkk)    \\
                                        &                                               \\
\textbf{Post 3}                         & \textbf{Post 6}                               \\
Kartofler!                              & S-Huset, fredagsbarer og joints               \\
Ansvarlig: \Hyttebombz                  & Ansvarlig: \Gabriel                           \\
Sted: Terrassen/køkken                  & Sted: Hemsen                                  \\
\end{tabular}

\makebox[\textwidth][c]{
\begin{tabular}{ | l | l | l | l | l | l | l | }
\hline
	Kl        & Post 1       & Post 2        & Post 3        & Post 4        & Post 5        & Post 6 \\ \hline
	15:00     & \Hippier     & \Bad          & \Fransk       & \Norder       & \Poppere      & \Alternative \\ \hline
	15:10     & \Alternative & \Hippier      & \Bad          & \Fransk       & \Norder       & \Poppere \\ \hline
	15:20     & \Poppere     & \Alternative  & \Hippier      & \Bad          & \Fransk       & \Norder \\ \hline
	15:30     & \Norder      & \Poppere      & \Alternative  & \Hippier      & \Bad          & \Fransk \\ \hline
	15:40     & \Fransk      & \Norder       & \Poppere      & \Alternative  & \Hippier      & \Bad \\ \hline
	15:50     & \Bad         & \Fransk       & \Norder       & \Poppere      & \Alternative  & \Hippier \\ \hline
\end{tabular}
}

\subsection{Poster}
\subsubsection{Post 1}
Hvor mange har allerede meldt sig ind i PF? \#GodBeslutning

\begin{itemize}
 \item En forening af de studerende, for de studerende.
 \item En 3-dimensionel forening, der engagerer sig både Fagligt, Politisk, Socialt og på alle områder imellem.
\end{itemize}

\begin{itemize}
 \item Fagligt
 \begin{itemize}
  \item Institutstudienævn - Evaluerer kurser og undervisere, for at sikre en høj kvalitet af undervisningen.
 \end{itemize}
 \item Politisk
 \begin{itemize}
  \item Uddannelsespolitisk råd - Sidder med i blandt andet DTU's bestyrelse, og deltager i diskussioner om uddannelsespolitik, der vedrører studerende på DTU.
 \end{itemize}
 \item Socialt
 \begin{itemize}
  \item S-Huset - Mere information hos \Gabriel.
  \item Idræt og klubber - Fodbold, Badminton, Basket, Volley, Løb, Bordtennis, Dans, Klatring, Sejlads, Ølbowling, Brætspil, Foto, Poker etc. Få støtte til at oprette en ny PF klub (PF Camping?). I kommer til at møde klubberne på en rundvisning i semsteruge 3.
 \end{itemize}
 \item Rabatter
 \begin{itemize}
  \item Ulykkesforsikring, Fitness World, Lyngby Svømmehal, Briller, Aviser og meget mere...
 \end{itemize}
\end{itemize}

Aktiv i de faglige råd.

Indmelding: www.pf.dk

Spørgsmål?

\subsubsection{Post 2}
Indstilling til kollegierne sker gennem Polyteknisk Forenings IndstillingsUdvalg (PFIU).\\
Ansøgning sker gennem www.pks.nu. Fortæl lidt om de forskellige kollegier og deres barer.
\begin{multicols}{2}
\begin{itemize}
 \item Kampsaxkollegiet - Saxen (Torsdag)
 \item Andelskollegiet - placering
 \item Prof. Ostenfeldt - Nakkeosten (1. tirsdag i måneden)
 \item Willum Kann Rasmussen - VKR-Baren (Mandag, ej lovlig)
 \item William Demant - Willys vandhul (Onsdag)
 \item P.O. Pedersen - Falladen (Tirsdag og lørdag)
 \item Paul Bergsøe - Pauls Ølstue (Tirsdag og fredag)
 \item Trørød (Par og unge med børn)
 \item Nybrogård - Kældercafeen (Fredag + evt. temafest om lørdagen) - Søges gennem KAB og ikke PKS
 \item Viggo Jarls -  - Søges gennem send mail til efor@kvjf.dk.
\end{itemize}
\end{multicols}
Det meste information kan findes i rusbogen. Priser fra 2250-3300 kr.

\textbf{Udlandsophold:}\\
Rigtig mange vælger at tage et semester eller to i udlandet. Det er fedt.
\begin{itemize}
\item Skal søges et halvt til et helt år før afhængig af destination
\item DTU har aftaler med en række universiteter og sørger for SU, bolig og indskrivning hos disse
\item Søger man andre universiteter kan det gøres hos Erasmus Fonden? eller andre organisationer, så skal man selv klare ovenstående
\item Der er rigtig mange legater man kan søge - også selvom man får SU
\item Man kan få hjælp hos International Affairs i administrationen i 101
\item Praktikophold kan søges hos IAESTE (International Association for the Exchange of Students for Technical Experience)
\end{itemize}

\subsubsection{Post 3}
Skræl de kartofler!

\subsubsection{Post 4}
Afdelingen for Uddannelse og Studerende befinder sig i 101A. Alle spørgsmål om SU, dispensationer og andet kan stilles her. Desuden kan studievejledningen hjælpe med tekniske detaljer om merit, studieforløb, retningsskifte, og meget andet.
KKO, Studenterrådgivningen og Studiepræsten.

\textbf{Fremdiftsreform}\\
\begin{itemize}
  \item{Tilmelding til fag} Alle studerende \emph{skal} være tilmeldt kurser svarende til 60 \emph{nye(!)} ECTS-point hvert år (2 x 13-ugers + 3-ugers). Normalvis 30 pr. semester.
  \item{Tilmelding til prøver} ALLE studerende tilmeldes automatisk eksaminer i de fag, de er tilmeldt. Så snart eftertilmeldingsperioden er slut, kan man \emph{ikke} vælge fag om, og er i udgangspunktet forpligtet til at bestå eksamen. Hvis man dumper, skal man til reeksamen så snart man kan. Enten næste semester eller i en særlig reeksamensperiode.
  \item{Studiestartsprøve} Inden for de to første måneder afholdes en prøve som \emph{skal} bestås for at kunne fortsætte på sit studie. For jer er prøven, at I skal lægge en studieplan for hele jeres bachelor. Mere har vi ikke fået at vide.
\end{itemize}

\subsubsection*{SU}
\label{sub:SU}
\begin{itemize}
  \item{Støttetidsregler} Hvis man søger ind på en videregående uddannelse for første gang senest to år efter endt gymnasiel uddannelse, har man 12 SU-klip ud over normeret studietid. Hvis man starter mere end to år efter endt gymnasielt stuie, er man kun berettiget til SU på normeret studietid.
  \item{SU-stop} Hidtil har man kunnet være 12 måneder bagefter med studiet, før SU'en inddrages. Disse regler ændres til 6 måneder, men træder først i kraft pr. \emph{1. september 2016}.
\end{itemize}

\subsubsection{Post 5}
Dans for helvede! DAAAAAAANS!!! (Reklamér for Rusjoint).

\subsubsection{Post 6}
S-huset er placeret i bygning 101 og åbner alle hverdag kl. 7.30, senere på aftenen rykker festen fra s-huset til Kælderbaren (s-huset lukker). Kælderbaren lukker når festen er gået kold (dog senest kl. 5.00). \#HuskS-HusetPåBallerup

\textbf{Kaffestuen:} mad og drikke. Special pris på bl.a. kaffe til PF medlemmer. Åben: Man-Fre: 07:30–19:00\\
\textbf{Pejsestuen:} Hyggeområde; sofaer, poolbord og bordfodboldbord.\\
\textbf{Læsesalen:} Bagerste lokale i S-huset. Brug biblio i stedet? God til eksamensforberedelse.\\
\textbf{Kælderbaren:} Specialøl, \#Awesomeness Åben: 19:00-?? (senest 05:00)\\

\textbf{PF Caféen:} S-huset består også af PF-Caféen i bygning 306. Lige overfor der hvor I har Mat 1. Åben: Man-Fre: 07:30 – 17:00\\

\textbf{Fredagsbarer}: Åbningstider 12:00-21:00 (eller 12:00-03:00 ved lang åbning). Hegnet åbner først 14:00 pga. Kantinen.

\begin{multicols}{2}
\begin{itemize}
  \item Diagonalen - Bygn. 116
  \item Etherrummet - Bygn. 208
  \item Hegnet - Bygn. 342
  \item Maskinen - Bygn. 358
  \item Diamanten - Bygn. 414
  \item Verners Kælder - Ballerup
\end{itemize}
\end{multicols}

\textbf{Joints:}
\begin{multicols}{2}
\begin{itemize}
  \item Sensommerfest - 4. september
  \item Rusjoint - 11. september (Spiser sammen i tværgrupper - billetter ryger hurtigt)
  \item Oktoberfest - 2-3. oktober (ikke sikker)
  \item Seriøst motionsløb - 8. oktober
  \item Useriøst motionsløb - 9. oktober
  \item Julejoint - ?
  \item Julefrokost - 7. november %opdateret dato
\end{itemize}
\end{multicols}