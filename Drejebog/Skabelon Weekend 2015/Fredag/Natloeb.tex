\section{Natløb}
\begin{tabular}{L{8cm} L{8cm}}
\textbf{Menneske Torpedo}           & \textbf{Jorden er giftig - Mælkekasser}   \\
Antal: 2 hold                       & Antal: 1 hold                             \\
Ansvarlig: \Gabriel                 & Ansvarlig: \Johnny                        \\
                                    &                                           \\
\textbf{Sæd-stafet}                 & \textbf{Pik mod drøbel (kropsdele på jord)}\\
Antal: 2 hold                       & Antal: 1 hold                              \\
Ansvarlig: \Lucyfar                        & Ansvarlig: \KABS                     \\
                                    &                                             \\
\textbf{Død stjernefisk}            & \textbf{Futtog - Mad/øl - Køkkenet}         \\
Antal: 1 hold                       & Antal: 1 hold                                 \\
Ansvarlig: \Ora                     & Ansvarlig: \Hyttebombz{}                       \\
\end{tabular}

\makebox[\textwidth][c]{
\begin{tabular}{ | c | c | c | c | c | c | c |   }
\hline
	Tid & Post 1 & Post 2 & Post 3 & Post 4 & Post 5 & Post 6 \\ \hline
	21:00 & \doublecell{\Hippier \\ \Norder   }  &  & \Fransk  & \Alternative & \Bad & \Poppere \\ \hline
	21:15 & \doublecell{\Bad \\ \Poppere}        &  & \Norder  & \Hippier & \Fransk & \Alternative \\ \hline
	21:30 & \doublecell{\Alternative \\ \Fransk} &  & \Poppere & \Bad & \Norder & \Hippier \\ \hline 
	21:45 &  & \doublecell{\Hippier \\ \Poppere} & \Bad        & \Norder & \Alternative & \Fransk \\ \hline
	22:00 &  & \doublecell{\Norder \\ \Fransk}   & \Alternative & \Poppere & \Hippier & \Bad \\ \hline
	22:15 &  & \doublecell{\Bad \\ \Alternative} & \Hippier    & \Fransk & \Poppere & \Norder \\ \hline
\end{tabular}
}

\subsection{Pointskema}
\begin{tabular}{ | l | l | l | l | l | l | }
\hline
	 & Samarbejde & Engagement & Awesomeness & Præstation & Bestikkelse \\ \hline
	\Hippier &  &  &  &  &  \\ \hline
	\Norder &  &  &  &  &  \\ \hline
	\Bad &  &  &  &  &  \\ \hline
	\Poppere &  &  &  &  &  \\ \hline
	\Alternative &  &  &  &  &  \\ \hline
	\Fransk &  &  &  &  &  \\ \hline
\end{tabular}

\newpage

\subsection{Poster}
\subsubsection{Post 1 - Menneske torpedo}
Ansvarlig:	\Gabriel eller \Lucyfar\\
Sted: Bakken \\
Antal: 2 hold \\\\
\textbf{Historie:}\\
Starfish High brænder, så du og dine med studerende skal ud af bygningen hurtigst muligt. Men på vejen er der nogle flasker med ekstra ilt der skal væltes. Den bagerste person er den eneste der kan se, og skal således guide de andre russer ud af bygningen, uden at kommunikere med dem mundtligt.\\\\
\textbf{Legen:}
Russerne stiller sig på en lang række, og holder hinanden på skuldrene. De forreste for bind for øjnene, så kun den bagerste person i rækken kan se. Denne person guider nu, med slag på skuldrene de andre russer i retning af ”ilt-flaskerne”, således at de vælter.\\\\
Der gives point for antallet af væltede ``ilt-flasker'', samarbejdet, kreativiteten samt bestikkelse. Og ja, det kommer til at tælle med i den endelige karakter.\\\\
\textbf{Materialer:}

\begin{multicols}{2}
\begin{itemize}
\item 10 tomme ølflasker
\item 20 bind (til øjnene)
\end{itemize}
\end{multicols}

\textbf{Rotationsskema for post 1:}\\
\begin{tabular}{ | l | }
\hline
	 \Hippier sendes videre til post 4, \Johnny på parkeringspladsen \\ \hline
	 \Norder sendes videre til post 3, \Ora på terrassen \\ \hline
	 \Bad sendes videre til post 4, \Johnny på parkeringspladsen \\ \hline
	 \Poppere sendes videre til post 3, \Ora på terrassen \\ \hline
	 \Alternative sendes videre til post 5, \KABS i Pejsestuen \\ \hline
	 \Fransk sendes videre til post 6, \Hyttebombz{} i køkkenet \\ \hline
\end{tabular}

\newpage

\subsubsection{Post 2 - Sæd-stafet}
Ansvarlig: 	\Lucyfar eller \Gabriel\\
Sted: Bakken \\
Antal: 2 hold \\\\
\textbf{Legen:}\\
I morgen er det \#PromNight og derfor vil det være en god idé at kunne håndtere forskellige typer væsker.
Ved I hvad denne type væske hedder \#LegMedSæden \#NåNej det er en newtonisk væske... \#DTU$<$3.

Sædstafetten går ud på at transportere alt sæden fra den ene skål til den anden vha. nogle plastickrus. Man skal naturligvis bruge munden \#Høhø, og man skal så hælde sæden fra den ene kop til den næste indtil man når den anden skål. Det går ud på at tømme sin egen skål så hurtigt som muligt.\\\\
Der gives point for stil, engagement, præstation og awesomeness. Og ja, det kommer til at tælle med i den endelige karakter.\\\\
\textbf{Materialer:}
\begin{multicols}{2}
\begin{itemize}
\item 4 gryder/skåle
\item 100 plastiskkrus
\item Vand i store mængder
\item 2 x kartoffelmel
\item Stor ske
\end{itemize}
\end{multicols}

\textbf{Rotationsskema for post 2:}\\
\begin{tabular}{ | l | }
\hline
	 \Hippier sendes videre til post 5, \KABS i Pejsestuen \\ \hline
	 \Poppere sendes videre til post 4, \Johnny på parkeringspladsen \\ \hline
	 \Norder sendes videre til post 6, \Hyttebombz{} i køkkenet \\ \hline
	 \Fransk sendes videre til post 4, \Johnny på parkeringspladsen \\ \hline
\end{tabular}



\subsubsection{Post 3 - Død stjernefisk}
Ansvarlig: \Ora \\
Sted: Terrasse\\
Antal: 1 hold\\\\
\textbf{Historie:}\\
Idrætshallens fundament er sunket grundet dårligt ingeniørarbejde, er øl, cider og sodavand fra opbevaringsrummet faldet ned, og dette må under ingen omstændigheder gå til spilde. Derfor skal I nu drikke alt det i overhovedet kan, uden at falde ned i hullet. \\\\
\textbf{Legen:}\\
Et afskærmet område 1x2 meter med en skål placeret i midten illustrer et hul i fundamentet,
Russerne vælger nu en der får et sugerør og skal drikke så meget som muligt fra skålen, uden
at røre området. Hjælperne holder russen, og ingen må betræde området. \\\\
Der giver point for antal flasker der drikkes, samarbejdet, kreativiteten og bestikkelse. 
Og ja, det kommer til at tælle med i den endelige karakter.\\\\
\textbf{Materialer:}
\begin{multicols}{2}
\begin{itemize}
\item 1 skål
\item 100 sugerør
\item Gaffetape
\item Vand
\end{itemize}
\end{multicols}

\textbf{Rotationsskema for post 3:}\\
\begin{tabular}{ | l | }
\hline
	 \Fransk sendes videre til post 5, \KABS i Pejsestuen \\ \hline
	 \Norder sendes videre til post 5, \KABS i Pejsestuen \\ \hline
	 \Poppere sendes videre til post 2, \Lucyfar og \Gabriel på bakken \\ \hline
	 \Bad sendes videre til post 6, \Hyttebombz{} i køkkenet \\ \hline
	 \Alternative sendes videre til post 2, \Lucyfar og \Gabriel på bakken \\ \hline
\end{tabular}



\subsubsection{Post 4 - Jorden er giftig (Mælkekasser)}
Ansvarlig:	\Johnny \\
Sted: Parkeringsplads \\
Antal: 1 hold \\\\
\textbf{Legen:}\\
Det er snart prom night, og da vi er landet ude på ødemarken i stedet for på vores fine skole, kan I risikere, at I skal igennem en masse mudder og græs for at komme til festsalen, og vi kan jo ikke have, at det fine tøj bliver ødelagt. Derfor skal I nu øve jer på at bevæge jer uden at røre jorden. Vi har de her mælkekasser, som vi skal bruge, og det er vigtigt, at alle kommer med.
Derfor skal I gå en rute (rundt om en tom ølflaske eller forbi et kendetegn eller noget) uden at røre jorden. \\\\
Der bliver givet point for stil, engagement, præstation og awesomeness. Og ja, det kommer til at tælle med i den endelige karakter.\\\\
\textbf{Materialer:}
\begin{multicols}{2}
\begin{itemize}
\item 4 mælke-/ølkasser
\item \#EnKæmpeFest!!!
\end{itemize}
\end{multicols}

\textbf{Rotationsskema for post 4:}\\
\begin{tabular}{ | l | }
\hline
	 \Alternative sendes videre til post 6, \Hyttebombz{} i køkkenet \\ \hline
	 \Hippier sendes videre til post 6, \Hyttebombz{} i køkkenet \\ \hline
	 \Bad sendes videre til post 3, \Ora på terrassen \\ \hline
	 \Norder sendes videre til post 2, \Lucyfar og \Gabriel på bakken \\ \hline
	 \Poppere sendes videre til post 5, \KABS i Pejsestuen \\ \hline
\end{tabular}



\subsubsection{Post 5 - Pik mod drøbel (kropsdel på jord)}
Ansvarlig: \KABS \\
Sted: Pejsestuen \\
Antal: 1 hold \\\\
\textbf{Legen:}\\
Nu er det \#LateNight idrætstime, og nattens tema er at lære at strække sig og stå i usædvanlige positioner for at styrke de statiske muskler og led. Måden det foregår på er, at jeg trækker et kort, som symboliserer noget, I skal gøre, fx øre mod mave, og så skal i stå sådan med kortet mellem de kropsdele, der bliver nævnt. Hvis I taber kortet eller slipper positionen, har I tabt.\\\\
Der bliver givet point for stil, engagement, præstation og awesomeness. Og ja, det kommer til at tælle med i den endelige karakter.\\\\
\textbf{Materialer:}
\begin{multicols}{2}
\begin{itemize}
  \item Kortspil
\end{itemize}
\end{multicols}

\textbf{Regler:}\\
\begin{tabular}{ | l | l | l | }
\hline
	Kort nr. & Kropsdel 1. person & Kropsdel 2. person \\ \hline
	Es & Hånd & Hånd \\ \hline
	2 & Røv & Røv \\ \hline
	3 & Hoved & Mave \\ \hline
	4 & Hånd & Røv \\ \hline
	5 & Albue & Skulder \\ \hline
	6 & Hånd & Hår \\ \hline
	7 & Kind & Skulder \\ \hline
	8 & Pande & Pande \\ \hline
	9 & Næse & Bryst \\ \hline
	10 & Mund & Hånd \\ \hline
	Knægt & Mund & Kind \\ \hline
	Dame & Lår & Røv \\ \hline
	Konge & Fod & Øre \\ \hline
\end{tabular}

\textbf{Rotationsskema for post 5:}\\
\begin{tabular}{ | l | }
\hline
	 \Bad sendes videre til post 1, \Lucyfar og \Gabriel på bakken \\ \hline
	 \Fransk sendes videre til post 1, \Lucyfar og \Gabriel på bakken \\ \hline
	 \Norder sendes videre til post 4, \Johnny på parkeringspladsen \\ \hline
	 \Alternative sendes videre til post 3, \Ora på terrassen \\ \hline
	 \Hippier sendes videre til post 3, \Ora på terrassen \\ \hline
\end{tabular}

\newpage

\subsubsection{Post 6 - Futtog (Køkken)}
Ansvarlig: \Hyttebombz{}\\
Sted: Køkkenet\\
Antal: 1 hold\\\\
\textbf{Materialer:}
\begin{multicols}{2}
\begin{itemize}
  \item 6 roulader
  \item 6 agurker \#FlereAgurker
  \item 6 liter koldskål
  \item 6 auberginer
  \item 6 liter (billig) kakaomælk
\end{itemize}
\end{multicols}

\textbf{Rotationsskema for post 6:}\\
\begin{tabular}{ | l | }
\hline
	 \Poppere sendes videre til post 1, \Lucyfar og \Gabriel på bakken \\ \hline
	 \Alternative sendes videre til post 1, \Lucyfar og \Gabriel på bakken \\ \hline
	 \Hippier sendes videre til post 2, \Lucyfar og \Gabriel på bakken \\ \hline
	 \Fransk  sendes videre til post 2, \Lucyfar og \Gabriel på bakken \\ \hline
	 \Bad sendes videre til post 2, \Lucyfar og \Gabriel på bakken \\ \hline
\end{tabular}



\subsection{Samlet materiale liste for natløb}
\begin{multicols}{2}
\begin{itemize}
% Post 1
\item 10 tomme ølflasker
\item 20 bind (til øjnene)
% Post 2+3
\item 4+1 gryder/skåle
\item 100 plastiskkrus
\item Vand i store mængder
\item 2 x kartoffelmel
% Post 3
%1 skål
\item 100 sugerør
% Post 4
\item 4 mælke-/ølkasser
% Post 5
\item Kortspil
% Post 6
\item 6 roulader
\item 6 agurker \#FlereAgurker
\item 6 liter koldskål
\item 6 auberginer
\item 6 liter (billig) kakaomælk
\end{itemize}
\end{multicols}